% This is a very simple Latex example
%
% This first bit is commented out

\documentclass[a4paper,11pt]{article}

\usepackage{epsf,epsfig,amsfonts} % allows epsfig comands
%\usepackage {html} % allows html links to be added
\usepackage{multicol}  % allows use of multi-columns within a page
\usepackage{times}     % space-saving font
\usepackage{color}     % allows colour fonts
\usepackage{bm}

\textheight 25.0cm
\textwidth 16cm
\topmargin -1.5cm
\oddsidemargin -0cm


%\pagestyle{plain}

% These are some ``macro'' defintions

\def\apj{{ApJ.}}
\def\apjs{{ApJS.}}
\def\apjl{{ApJL.}}
\def\nat{{Nature.}}
\def\mnras{{MNRAS}}
\def\araa{{ARA\&A}}
\def\aap{{A\&A}}
\def\aj{{AJ}}
\def\aaps{{A\&AS}}
\def\apss{{Ap\&SS}}

\def\spose#1{\hbox to 0pt{#1\hss}}
\def\simlt{\mathrel{\spose{\lower 3pt\hbox{$\mathchar"218$}}
     \raise 2.0pt\hbox{$\mathchar"13C$}}}
\def\simgt{\mathrel{\spose{\lower 3pt\hbox{$\mathchar"218$}}
     \raise 2.0pt\hbox{$\mathchar"13E$}}}

% itemz environment: Like itemize but waste less space
 \newenvironment{itemz}
 {\begin{list}{$\bullet$}{\setlength{\itemsep}{0pt}}}
 {\end{list}}



% Setting up some stuff that will go into the title
\author{Peter Hurley, Seb Oliver}
\title{HELP: XID+}
\date{July 11th 2014}

% Now we start doing some stuff

\begin{document}

% This is where the title is written
\maketitle

\tableofcontents
%\listoffigures
%\listoftables


\section{Introduction}
\section{Posterior Predictive Checking}
All models are wrong. What we would like to know is what aspects of the model don't fit the data. If our models are  for the SPIRE maps are reasonable, the observed SPIRE maps should look plausible under the posterior predictive distribution. To asses the model fits provided by XID+, we compare the predictive distribution $\mathbf{y}^{\mathrm{rep}}$ to the observed data $\mathbf{y}$. Unlike frequentist statistics, posterior predictive checking takes into account the uncertainty associated with the estimated parameters of the model. As posterior predictive checks involve a double use of the data, usage should be limited to measuring discrepancy to study model adequacy and not model comparison or inference (Meng 1994).

Discrepency between model and data is measured via a test quantity or discrepancy measure, $T(\mathbf{y},\mathbf{\Theta})$,  a scaler summary of all parameters and data. A test statistic  only depends on data. In Bayesian context, test statistics can be generalised to allow dependence on model parameters under posterior distribution.

\subsection{Tail area probabilities (or p-value)} 
Classical p-value for the test statistic $T(\mathbf{y})$ is
\begin{equation}
p_{C} = \Pr(T(\mathbf{y}^{\mathrm{rep}})\le T(\mathbf{y})|\mathbf{\Theta})
\end{equation}
where $\Theta$ is fixed. Test statistics represent a summary measure of discrepancy between observed data and what would be expected under a model with a particular value of $\mathbf{Theta}$ such as the maximum likelihood value. 

Possible test statistics for SPIRE maps include:
\begin{itemize}
\item $\mathrm{E}(\mathbf{y})$
\end{itemize}

\subsection{Graphical posterior predictive checks}
Basic idea is to display data alongside simulated data and look for systematic discrepancies. This could include:
\begin{itemize}
\item direct display of data i.e. map and simulated maps. as in BDA, ordering of pixels could be important (though is the case for maps?) Could also display stack of sources (mean and/or median)
\item display of data summaries (i.e. histogram of pixel flux densities) or parameter inferences (I don't understand the example in BDA for how parameter inferences can help).
\item display of residuals, though they are in Bayesian in this context as they include uncertainty in parameters. Histogram of residuals might also be useful.
\end{itemize}

The Lacey simulations seem to underestimate the flux for sources and the residual histogram suggests too much flux in the model. This could be a result of putting too much flux in the faint sources due to uniform prior. I need to come up with proper graphical posterior predictive checks. 


I need a measure of when sources flux is being incorrectly estimated according to data. 

Bias is far more complicated. Bayesian models are inherently biased by nature. To get a handle on bias, best way is to show what happens when you input sources with zero flux (or close to) 
 
\end{document}



