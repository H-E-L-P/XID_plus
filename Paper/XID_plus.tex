% mn2eguide.tex
% v2.1 released 03/05/2002
%
% Adapted from mnguide.tex
% v1.3 released 14th September 1995
% v1.2 released 5th September 1994 (M. Reed)
% v1.1 released 18th July 1994
% v1.0 released 28th January 1994


% The journal style files and macros, with guides on their use, are
% available by anonymous FTP on the Internet from the Comprehensive
% TeX Archive Network (CTAN) sites ftp.tex.ac.uk and ftp.dante.de.
% The files are in the directories
% /tex-archive/macros/plain/contrib/mnras and
% /tex-archive/macros/latex209/contrib/mnras for the TeX and LaTeX
% files respectively.



\documentclass[useAMS,usenatbib]{mn2e}

\usepackage{rotating}
\usepackage{lscape}
\usepackage{graphicx}
\usepackage{amssymb}
\usepackage{amsmath}
\usepackage{epstopdf}
\usepackage{color}
\usepackage{soul}
\usepackage{multirow}
\usepackage{longtable}
\usepackage{textcomp}
\usepackage[caption=false]{subfig}
\usepackage{float}
\usepackage{appendix}


\def\mnras{MNRAS}
 \def\apj{Astrophys. J.}
 \def\aap{Astron. Astrophys.}
 \def\apjs{Astrophys. J., Suppl. Ser.}



%--------------------------------------------------------
\bibliographystyle{mn2eNicola}

\usepackage{hyperref}

\title[HELP:XID+]
  {HELP: Probabilistic De-blender for Herschel SPIRE maps}\author[P.D. Hurley et al.]{P.D.~Hurley,$^1$\thanks{Email: p.d.hurley@sussex.ac.uk} S.~Oliver,$^1$\\
$^1$Astronomy Centre, Department of Physics and Astronomy, University of Sussex, Falmer, Brighton BN1 9QH, UK\\}

\date{Released 2002 Xxxxx XX}

\pagerange{\pageref{firstpage}--\pageref{lastpage}} \pubyear{2002}

\def\LaTeX{L\kern-.36em\raise.3ex\hbox{a}\kern-.15em
    T\kern-.1667em\lower.7ex\hbox{E}\kern-.125emX}

\newtheorem{theorem}{Theorem}[section]
\graphicspath{}
\begin{document}

\label{firstpage}
\maketitle

\begin{abstract}
The Herschel Extragalactic Legacy Project (HELP) will provide ancillary data from other wavelengths alongside the extra
Software available at \url{https://github.com/pdh21/XID_plus/}.
\end{abstract}


\begin{keywords}
galaxies: statistics -- infrared: galaxies
\end{keywords}
%
%
\section{Introduction}
\section{Data}
\section{XID+ Algorithm}
%probabilisitc programming, probabilistic models, inference on models... why do science with catalogues when we can do science with maps? 

\subsection{XID and DESPHOT}
 Although the XID+ algorithm presented in this paper takes different approaches, it builds upon knowledge gained from developing the original XID (a.k.a DESPHOT) algorithm used by HerMES \citep{Roseboom:2010, Roseboom:2011, Wang:2014}. Details of the algorithm can be found in the corresponding papers, but for convenience we list some of the main details. 
 
DESPHOT consists of following main steps: map segmentation, source photometry and noise estimation. 

\subsubsection{Map segmentation}
Ideally, source photometry and background estimation would be done on full image, in practice it is often computationally infeasible. DESPHOT segmented the map by locating islands of high SNR pixels enclosed by low SNR pixels. The segmentation algorithm operates thus:
\begin{itemize}
\item Locates all pixels with a SNR above some threshold (default value of SNR= 1);
\item Takes the first of these high SNR pixel starting in the bottom left corner of the image;
\item `Grows' a region around this pixel by iteratively taking neighbouring high SNR pixels;
\item Once there are no more high SNR neighbours jumps to the next high SNR pixel and repeat from step (iii).
Each of these independent regions of high SNR pixels is uniquely identified and processed separately by the source photometry component.
\end{itemize}
\subsubsection{Source photometry}
DESPHOT assumes the map can be described by the sum of the flux densities from $n$ known sources, a global background level and some unknown noise term

\begin{equation}
\mathbf{d} = \sum\limits_{i=1}^n \mathbf{P_i}f_i + B + \delta
\label{eq:map}
\end{equation}
where $\mathbf{d}$ is the image, $\mathbf{P_i}$ is the PRF for source $i$, $f_i$ is the flux density for source $i$, $B$ is a global background estimate and $\delta$ is the noise term. As this is a linear equation, it has a maximum likelihood solution which can be solved directly by matrix inversion or via other linear methods. As discussed in \cite{Roseboom:2010, Roseboom:2011, Wang:2014}, these approaches ignore prior knowledge that fluxes cannot have negative flux density, which in very degenerative cases can result in any symmetric pairing of positive and negative flux providing a good fit. They are also incapable of discriminating between real and spurious sources, which can result in overfitting. To overcome these issues, \cite{Roseboom:2011} used the non-negative weighted LASSO algorithm (Tibshirani 1996; Zou 2006; ter Braak et al 2010).

LASSO works by treating sources either `inactive' and flux density set to zero, or `active'. It switches sources on one at a time, with the order determined by reduction in chi-squared gained by turning them on. The process continues until some tolerance is reached.

In the first iteration, DESPHOT uses LASSO on each segment, to estimate the source fluxes. It then estimates a value for the background (B) via
\begin{equation}
B = \mathbf{d} - \sum\limits_{i=1}^n \mathbf{P_i}f_i
\end{equation} 

The estimate from B is subtracted, and the LASSO fitting is rerun to get the final flux density estimates.
\subsubsection{Noise estimation}
If DESPHOT is assumed to be linear\footnote{introducing LASSO and non-negative priors introduces a non-linearity}, then one can get a lower limit on the noise from $(\mathbf{A^TN_d^{-1}A})^{-1}$. This estimate only includes instrumental noise and degeneracies between sources. An estimate of the remaining residual confusion noise is is calculated by taking the standard deviation of the residual map pixels $\sigma{res}$ and removing the average instrumental noise in these pixels in quadrature, $\sigma^2_{conf} = \sigma^2_{res} - \sigma^2_{pix}$, where $\sigma_{pix}$ is calculated directly from the exposure time per pixel. The total noise $\sigma_{tot}$ for a point source is then calculated from both the instrumental noise (and confusion noise from the known sources), $\sigma_{i} = \sqrt{\mathrm{diag}((\mathbf{A^TN^{-1}_dA)^{-1})}}$, and confusion noise from the unknown sources in the residual map $\sigma_{conf}$ via $\sigma^2_{tot} = \sigma^2_{i} + \sigma^2_{conf}$. 
 
 
 One of the goals of HELP is to extend the use of prior information in order to get below the noise level introduced by source confusion.
\subsection{Probabilistic Model}
With XID+, we have adopted a Bayesian probabilistic framework to provide us with a natural and transparent way to include prior information. To use this probabilistic framework, we need to create a generative model for the maps. XID+ is the simplest example of a generative model for the Herschel SPIRE maps. Figure \ref{} shows our probabilistic graphical model for our basic XID+ model, where boxes represent dimensions, open circles as variables, dots as deterministic or (fixed) variables. For our simplest model, the sky co-ordinates of our sources are treated as fixed and along with the fixed PRF. Both these fixed variables are used to make the pointing matrix $A$ which details the contribution each source makes to each pixel $j$ in the map. Each source has its own flux $f$ which is a random variable. By combining $f$, $A$ and our global estimate for the background $B$, we can make our model map $m$   our model for the map ($M$)the map ($D$) however one can easily add additional details. 

%Our model for the map i.e. background, sources, prior flux information etc

\subsection{Stan}
%How Stan fits the map, chains, runtime, estimating convergence
\subsection{Segmentation}
%How we segment the map
\subsection{Uncertainties and Covariances}
%Details on what we get out of fit. e.g. marginalised estimate of fluxes, covariance information between sources
The uncertainties from the posterior give the uncertainty of the flux given the data. This includes the uncertainty from instrumental noise and confusion (I think). Unlike XID, we are not solving $\mathbf{f}=(\mathbf{A^TN_d^{-1}A})^{-1}\mathbf{A^TN_d^{-1}d}$, we are solving equation \ref{eq:map} and so variations in pixel flux from sources not in the prior list, i.e. from confusion, will directly affect our flux estimates.
\section{Simulations}
\subsection{XID+ vs. DESPHOT}
%testing agains DESPHOT
\subsection{Performance with Priors}
%Show how performance increases with better prior information. Though this is SPIRE PRIOR model challenge
Adding prior information on fluxes 
\section{XID+Science}
%Show how framework allows users to do science with maps 
\section{Conclusions}



\section*{Acknowledgements} %
%
%%
%
%
\bibliography{}

%
%
%
%
%
%%%%%%%%%%%%%%%%%%%%%%%%%%%%%%%%%%%%%
%
%% \bsp % ``This paper has been produced using the ...''
%
%\label{lastpage}

\end{document}
