% mn2eguide.tex
% v2.1 released 03/05/2002
%
% Adapted from mnguide.tex
% v1.3 released 14th September 1995
% v1.2 released 5th September 1994 (M. Reed)
% v1.1 released 18th July 1994
% v1.0 released 28th January 1994


% The journal style files and macros, with guides on their use, are
% available by anonymous FTP on the Internet from the Comprehensive
% TeX Archive Network (CTAN) sites ftp.tex.ac.uk and ftp.dante.de.
% The files are in the directories
% /tex-archive/macros/plain/contrib/mnras and
% /tex-archive/macros/latex209/contrib/mnras for the TeX and LaTeX
% files respectively.



\documentclass[useAMS,usenatbib]{mn2e}

\usepackage{rotating}
\usepackage{lscape}
\usepackage{graphicx}
\usepackage{amssymb}
\usepackage{amsmath}
\usepackage{epstopdf}
\usepackage{color}
\usepackage{soul}
\usepackage{multirow}
\usepackage{longtable}
\usepackage{textcomp}
\usepackage[caption=false]{subfig}
\usepackage{float}
\usepackage{appendix}


\def\mnras{MNRAS}
 \def\apj{Astrophys. J.}
 \def\aap{Astron. Astrophys.}
 \def\apjs{Astrophys. J., Suppl. Ser.}



%--------------------------------------------------------
\bibliographystyle{mn2eNicola}

\usepackage{hyperref}

\title[HELP:XID+]
  {HELP: Probabilistic De-blender for Herschel SPIRE maps}\author[P.D. Hurley et al.]{P.D.~Hurley,$^1$\thanks{Email: p.d.hurley@sussex.ac.uk} S.~Oliver,$^1$\\
$^1$Astronomy Centre, Department of Physics and Astronomy, University of Sussex, Falmer, Brighton BN1 9QH, UK\\}

\date{Released 2002 Xxxxx XX}

\pagerange{\pageref{firstpage}--\pageref{lastpage}} \pubyear{2002}

\def\LaTeX{L\kern-.36em\raise.3ex\hbox{a}\kern-.15em
    T\kern-.1667em\lower.7ex\hbox{E}\kern-.125emX}

\newtheorem{theorem}{Theorem}[section]
\graphicspath{}
\begin{document}

\label{firstpage}
\maketitle

\begin{abstract}
The Herschel Extragalactic Legacy Project (HELP) will provide ancillary data from other wavelengths alongside the extra
Software available at \url{https://github.com/pdh21/XID_plus/}.
\end{abstract}


\begin{keywords}
galaxies: statistics -- infrared: galaxies
\end{keywords}
%
%
\section{Introduction}
\section{Data}
\section{Algorithm}
\subsection{Probabilistic Model}
%Our model for the map i.e. background, sources, prior flux information etc
\subsection{Stan}
%How Stan fits the map, chains, runtime, estimating convergence
\subsection{Segmentation}
%How we segment the map
\subsection{Uncertainties and Covariances}
%Details on what we get out of fit. e.g. marginalised estimate of fluxes, covariance information between sources
\section{Simulations}
\subsection{XID+ vs. DESPHOT}
%testing agains DESPHOT
\subsection{Performance with Priors}
%Show how performance increases with better prior information
\section{XID+Science}
%Show how framework allows users to do science with maps 
\section{Conclusions}



\section*{Acknowledgements} %
%
%%
%
%
\bibliography{}

%
%
%
%
%
%%%%%%%%%%%%%%%%%%%%%%%%%%%%%%%%%%%%%
%
%% \bsp % ``This paper has been produced using the ...''
%
%\label{lastpage}

\end{document}
